\documentclass[10pt,landscape, a4paper]{article}
\usepackage[legalpaper, landscape, right=.2in, top=.5in, bottom=.5in, left=.2in]{geometry}
\usepackage{multicol}
\usepackage{amsfonts}
\usepackage{amsmath,amsthm}
\usepackage{amssymb}
\usepackage{graphicx}
\usepackage{tcolorbox}
\setlength{\columnseprule}{1pt}
\setcounter{section}{1}


\newcommand{\custombox}[3]{\begin{tcolorbox}[title = \textbf{#1}, colback=#2!10!white, colframe = #2!70!white, coltitle=white]
    #3
    \end{tcolorbox}}

\newcommand{\theorem}[2]{\custombox{Theorem #1}{red}{#2}}
\newcommand{\definition}[2]{\custombox{Definition #1}{orange}{#2}}
\newcommand{\prop}[2]{\custombox{Proposition #1}{black!50!orange!30!yellow}{#2}}
\newcommand{\other}[2]{\custombox{#1}{green!60!black}{#2}}
    


\begin{document}
\begin{multicols*}{3}
    \begin{center}
        \Large{\textbf{Analysis II HS21}} \\
        \small{by plaffranchi}
    \end{center}
    \section{ODE (ordinary differential equation)}
    \theorem{2.1.6}{Let $F: \mathbb{R}^{2} \rightarrow \mathbb{R}$ be differentiable. Let $x_{0} \in \mathbb{R}$ and $y_{0} \in \mathbb{R}^{2}$. Then the ODE
    $ y^{\prime}=F(x, y) $ has a unique solution $f$ defined on a "largest" open interval I containing $x_{0}$ such that $f\left(x_{0}\right)=y_{0} .$}
    \definition{2.2.1}{Let $I \subset \mathbb{R}$ be an open interval and $k \in \mathbb{N}_0$. An homogeneous linear ODE of order $k$ on $I$ is of the form
    $
        y^{(k)}+a_{k-1} y^{(k-1)}+\cdots+a_{1} y^{\prime}+a_{0} y=0
    $
    where the coefficients $a_{0}, \ldots, a_{k-1}$ are complex-valued functions on $I$, and the unknown is a function $I \to \mathbb{C}$ that is $k$-times differentiable on $I$.
    An equation of the form
    $
        y^{(k)}+a_{k-1} y^{(k-1)}+\cdots+a_{1} y^{\prime}+a_{0} y=b,
    $
    where $b: I \rightarrow \mathbb{C}$ is another function, is called an inhomogeneous linear ODE.}
    \other{Recognize an ODE}{\begin{enumerate}
        \item no coefficients before the highest derivative
        \item all coefficients are continuous
        \item no products of $y$ or their derivatives
        \item no powers of $y$ or their derivatives
        \item no functions depending on $y$ or their derivatives
    \end{enumerate}}

    % Theorem 2.2.3?


    \prop{2.3.1}{Any solution of $y'+ay=0$ is of the form $f(x)=z\exp(-A(x))$ where $A$ is a primitive of $a$. The unique solution with $f(x_0)=y_0$ is $f(x)= y_0\exp(A(x_0)-A(x))$.}

    \other{Solving inhomogeneous equations}{\textbf{Case 1:} Make a guess. For example $y'=y+x^2$ guess $f(x) = ax^2+bx+c$, and solve the equation.\\
    \textbf{Case 2:} Use the variation of the constant. Assume $f_p=z(x)\exp(-A(x))$ for $z: I\to \mathbb{C}$. Then $z'(x) = b(x)\exp(A(x)) \implies k(x) = \int b(x)\exp(A(x))dx$.}
    \definition{Linear differential equations with constant coefficients}{Let $k \in \mathbb{N}_0$, $a_0,...,a_{k-1} \in \mathbb{C}$ fixed and $b$ a general continuous function, then $y^{(k)}+a_{k-1} y^{(k-1)}+\cdots+a_{1} y^{\prime}+a_{0} y=b$ is such equation.}
    \other{Solution of hom. diff. eq. with constant coefficients}{Look for solutions of the form $f(x)=e^{\alpha x}$ for $\alpha \in \mathbb{C}$. Then we have $f^{(j)}(x)=\alpha^{j} e^{\alpha x}$ for all $j \geqslant 0$ and for all $x$, which means that
    \begin{align*}
        &f^{(k)}(x)+a_{k-1} f^{(k-1)}(x)+\cdots+a_{1} f^{\prime}(x)+a_{0} f(x)\\
        = &e^{\alpha x}\left(\alpha^{k}+a_{k-1} \alpha^{k-1}+\cdots+a_{1} \alpha+a_{0}\right) .
    \end{align*} }


EXAMPLE 2.4.1. The equation $y^{\prime \prime}-x y=0$ does not belong to this class, but the equations $y^{\prime}-y=0$ and $y^{\prime \prime}+y=0$ (satisfied by the exponential and by trigonometric functions) have constant coefficients.


We conclude that $f$ is a solution of the homogeneous equation if and only if $P(\alpha)=0$, where $P$ is the polynomial with coefficients $a_{0}, \ldots, a_{k-1}$ :
$$
P(X)=X^{k}+a_{k-1} X^{k}+\cdots+a_{1} X+a_{0} .
$$
According to the Fundamental Theorem of Algebra, this polynomial of degree $k$ has $k$ complex roots, counted with multiplicity: there exist complex numbers $\alpha_{1}, \ldots, \alpha_{k}$ such that
$$
P(X)=\left(X-\alpha_{1}\right) \cdots\left(X-\alpha_{k}\right) .
$$
This polynomial is called the companion or characteristic polynomial of the homogeneous differential equation.

REMARK 2.4.2. We repeat that this is only defined when the coefficients of the equation are constant.

REMARK 2.4.3. Although it is natural to look for complex-valued solutions, one is often interested in situations where the coefficients $a_{i}$ are real and we know that the solution should take real values, or we want such solutions.

Suppose that a root $\alpha=\beta+i \gamma$ is not real, so the imaginary part $\gamma$ is non-zero. Then the solution $f(x)=e^{\alpha x}$ does not take real values. However, in that case, the conjugate


\end{multicols*}
\end{document}