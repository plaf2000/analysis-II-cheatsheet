\other{Vector field example}{
    A vector field often discussed in the series and seen in some lecture's examples:
    $$f: \R^2\setminus \{(0,0)\} \to \R^2; f(x,y) = \left(\begin{matrix}
        \frac{-y}{x^2+y^2}\\
        \frac{x}{x^2+y^2}
    \end{matrix}\right).$$
    This vector field is \textbf{not} conservative, even if $\frac{\partial f_2}{\partial x}=\frac{\partial f_1}{\partial y}$. For example, the line integral over the parameterized curve of a "clockwise" closed circle of radius 1 doesn't give 0 as expected, but $\pi/2$ instead.\\
    We can though define the potentials:
    \begin{itemize}
        \item $g_1: \R^+\times \R; (x,y) \mapsto \arctan(y/x)$
        \item $g_2: \R\times \R^+; (x,y) \mapsto \arctan(-x/y) = \arctan(y/x) - \pi/2$
        \item $g_3: \R^-\times \R; (x,y) \mapsto \arctan(y/x)$
        \item $g_4: \R\times \R^-; (x,y) \mapsto \arctan(-x/y) = \arctan(y/x) - \pi/2$
    \end{itemize}
}

\other{Trigonometric functions}{
    \begin{center}
        \begin{tabular}{c|c|c|c|c|c|c|c|c|c}
             \(\alpha\)& 0& \(30^{\circ}\)& \(45^{\circ}\)& \(60^{\circ}\)& \(90^{\circ}\)& \(120^{\circ}\)& 150\(^{\circ}\)& 180\(^{\circ}\)
             \\
             \(\)& 0& \(\pi/6\)& \(\pi/4\)& \(\pi/3\)& \(\pi/2\)&\(2\pi/3\)& \(5\pi/6\)& \(\pi\) \\
             \hline
             \(\sin{\alpha}\)& 0& \(1/2\)& \(\sqrt{2}/{2}\)& \(\sqrt{3}/{2}\)& 1& \(\sqrt{3}/{2}\)& \(1/2\)& 0
             \\
             \hline
             \(\cos{\alpha}\)& 1& \(\sqrt{3}/{2}\)& \(\sqrt{2}/{2}\)& \(1/2\)& 0& \(-1/2\)& \(-\sqrt{3}/{2}\)& -1
             \\
             \hline
             \(\tan{\alpha}\)& 0& \(\sqrt{3}/{3}\)& \(1\)& \(\sqrt{3}\)& N/A & \(-\sqrt{3}\)& \(-\sqrt{3}/{3}\)& 0
             \\
        \end{tabular}
        \end{center}
}

\prop{Trigonometric properties (S3.42 + K3.43 Ana I)} {
    \begin{enumerate}[label=(\arabic*)]
        \item $\exp(iz)=\cos z+i\sin z$
        \item $\cos z = \cos(-z)$, $\sin(-z) = - \sin(z)$
        \item $\sin z = \frac{e^{iz}-e^{-iz}}{2i}$, $\cos z = \frac{e^{iz}+-e^{-iz}}{2}$
        \item $\sin(z + w) = \sin(z) \cos(w) + \cos(z) \sin(w)$\\
              $\implies \sin(2z)=2\sin z\cos z$,\\
              $\cos(z + w) = \cos(z) \cos(w) - \sin(z) \sin(w)$\\
              $\implies \cos(2z)=\cos^2z-\sin^2z$,
        \item $\cos^2z+\sin^2z=1$
    \end{enumerate}
}