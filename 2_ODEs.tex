\theorem{2.1.6}{
    Let $F: \mathbb{R}^{2} \rightarrow \mathbb{R}$ be differentiable. Let $x_{0} \in \mathbb{R}$ and $y_{0} \in \mathbb{R}^{2}$. Then the ODE
    $ y^{\prime}=F(x, y) $ has a \textbf{unique solution} $f$ defined on a "largest" open interval I containing $x_{0}$ such that $f\left(x_{0}\right)=y_{0} .$}
    \definition{2.2.1}{Let $I \subset \mathbb{R}$ be an open interval and $k \in \mathbb{N}_0$. An \textbf{homogeneous linear ODE of order} $k$ on $I$ is of the form
    $
        y^{(k)}+a_{k-1} y^{(k-1)}+\cdots+a_{1} y^{\prime}+a_{0} y=0
    $
    where the coefficients $a_{0}, \ldots, a_{k-1}$ are complex-valued functions on $I$, and the unknown is a function $I \to \mathbb{C}$ that is $k$-times differentiable on $I$.
    An equation of the form
    $
        y^{(k)}+a_{k-1} y^{(k-1)}+\cdots+a_{1} y^{\prime}+a_{0} y=b,
    $
    where $b: I \rightarrow \mathbb{C}$ is another function, is called an \textbf{inhomogeneous linear ODE}.}
    \other{Recognize a linear ODE}{\begin{enumerate}
        \item no coefficients before the highest derivative
        \item all coefficients are continuous
        \item no products of $y$ or their derivatives
        \item no powers of $y$ or their derivatives
        \item no functions depending on $y$ or their derivatives
    \end{enumerate}
}


\prop{2.3.1}{
    Any solution of $y'+ay=0$ is of the form $f(x)=z\exp(-A(x))$ where $A$ is a primitive of $a$. The unique solution with $f(x_0)=y_0$ is $f(x)= y_0\exp(A(x_0)-A(x))$.
}


\other{Solving inhomogeneous equations}{
    \textbf{Case 1:} Make a guess. For example $y'=y+x^2$ guess $f(x) = ax^2+bx+c$, and solve the equation.\\
    \textbf{Case 2:} Use the variation of the constant. Assume $f_p=z(x)\exp(-A(x))$ for $z: I\to \mathbb{C}$. Then $z'(x) = b(x)\exp(A(x)) \implies k(x) = \int b(x)\exp(A(x))dx$.
}


\definition{Linear differential equations with constant coefficients}{
    Let $k \in \mathbb{N}_0$, $a_0,...,a_{k-1} \in \mathbb{C}$ fixed and $b$ a general continuous function, then $y^{(k)}+a_{k-1} y^{(k-1)}+\cdots+a_{1} y^{\prime}+a_{0} y=b$ is such equation.
}


\other{Solution of hom. diff. eq. with constant coefficients}{
    Look for solutions of the form $f(x)=e^{\alpha x}$ for $\alpha \in \mathbb{C}$. Then we have $f^{(j)}(x)=\alpha^{j} e^{\alpha x}$ for all $j \geqslant 0$ and for all $x$, which means that
    \begin{align*}
        & f^{(k)}(x)+a_{k-1} f^{(k-1)}(x)+\cdots+a_{1} f^{\prime}(x)+a_{0} f(x)                \\
        = & e^{\alpha x}\left(\alpha^{k}+a_{k-1} \alpha^{k-1}+\cdots+a_{1} \alpha+a_{0}\right) .
    \end{align*}
    This translates into finding the zeros ($\alpha_1,...,\alpha_k \in \mathbb{C}$) of the characteristic polynomial:
    \begin{align*}
        P(X)= & X^{k}+a_{k-1} X^{k}+\cdots+a_{1} X+a_{0}                      \\
        =     & \left(X-\alpha_{1}\right) \cdots\left(X-\alpha_{k}\right) = 0
    \end{align*} 
}


\other{Imaginary roots}{
    If a root is not real i.e. $\alpha = \beta + i\gamma$, the solution $f(x)=e^{\alpha x}$ does not take real values, but $\overline{\alpha} = \beta - i\gamma$ is also a root, hence we can write $\widetilde{f}_{1}(x)=e^{\beta x} \cos (\gamma x), \quad \widetilde{f}_{2}(x)=e^{\beta x} \sin (\gamma x)$ instead of $f_{1}(x)=e^{\alpha x}, \quad f_{2}(x)=e^{\bar{\alpha} x}$
}


\other{Multiple roots}{
    \textbf{Case 1: no multiple roots.} Any solution of the equation is of the form $f(x) = z_1e^{a_1x}+\cdots +z_ke^{a_kx}$.\\
    \textbf{Case 2: multiple roots.}  Suppose that $\alpha$ is a multiple root of order $j$ with $2 \leqslant j \leqslant k$. Then the $k$ functions
    $
        f_{\alpha, 0}(x)=e^{\alpha x}, \quad f_{\alpha, 1}(x)=x e^{\alpha x}, \quad \cdots, \quad f_{\alpha, j-1}(x)=x^{j-1} e^{\alpha x}
    $
    are linearly independent solutions. Taking the union of the functions $f_{\alpha, j}$ for all roots of $P$, each with its multiplicity, gives a basis of the space of solutions.
}

\other{Special form solutions}{
    For $d\geqslant0$ and $Q,Q_1,Q_2$ are polynomials of degree $d$ if $\beta$ is a root of the companion polynomial, or $Q,Q_1,Q_2$ are polynomials of degree $d+j$ if $\beta$ is a root of the companion polynomial of multiplicity j.
\begin{center}
        \begin{tabular}{|c|c|} 
         \hline
          $b(x)$ & Solution's form \\ \hline
          $x^de^{\beta x}$ & $Q(x)e^{\beta x}$\\ \hline
          $x^d\cos(\beta x)$ & $Q_1(x)\cos(\beta x)+Q_2(x)\sin(\beta x)$\\ \hline
          $x^d\sin(\beta x)$ & $Q_1(x)\cos(\beta x)+Q_2(x)\sin(\beta x)$\\ \hline
        \end{tabular}
      \end{center}
}